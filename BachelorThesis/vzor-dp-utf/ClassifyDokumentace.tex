\documentclass[12pt, a4paper]{report}

\usepackage[czech]{babel}
\usepackage[utf8]{inputenc}
\usepackage[cp1250]{inputenc}
\usepackage[IL2]{fontenc}
\usepackage{graphicx}
\usepackage{float}


\title{Klasifikace textu do kategorií spam/nespam}
\author{David Pivovar A13B0403P}

\begin{document}

%%%%%Titulni strana%%%%%%%%%%%%%%%%%%%%%%%%%%%%%%%%%%%%%%%%%%%%%%%%%%%%%%
\begin{titlepage}

\begin{figure}[t]
\centering
	\includegraphics{logo_fav.jpg}
	\label{fig:logo_fav}
\end{figure}

\begin{center}
	{\large KIV/PC - Semestrální práce\\[0.3cm]}
	{\huge Klasifikace textu do kategorií spam/nespam \\[1.7cm] }
\end{center}

\vfill

\begin{flushleft}
	{\large \textbf{David Pivovar}\\}
	{\large A13B0403P}
	\hfill
	{\large \today}
\end{flushleft}

\end{titlepage}
%%%%%%%%%%%%%%%%%%%%%%%%%%%%%%%%%%%%%%%%%%%%%%%%%%%%%%%%%%%


%%%%%Obsah%%%%%%%%%%%%%%%%%%%%%%%%%%%%%%%%%%%%%%%%%%%%%%%%%
\tableofcontents
%%%%%%%%%%%%%%%%%%%%%%%%%%%%%%%%%%%%%%%%%%%%%%%%%%%%%%%%%%%


%%%%%Zadani%%%%%%%%%%%%%%%%%%%%%%%%%%%%%%%%%%%%%%%%%%%%%%%%
\chapter{Zadání}

Naprogramujte v ANSI C přenositelnou konzolovou aplikaci, která rozhodne zda úsek textu
je nebo není spam.\\
\\
Aplikace bude přijímat z příkazové řádky sedm parametrů: První dva parametry budou vzor jména a
počet trénovacích souborů obsahujících nevyžádané zprávy (spam).
Třetí a čtvrtý parametr budou vzor jména a počet trénovacích souborů obsahujících vyžádané zprávy
(ham).
Pátý a šestý parametr budou vzor jména a počet testovacích souborů.
Sedmý parametr představuje jméno výstupního textového souboru, který bude po dokončení činnosti
Vaší aplikace obsahovat výsledky klasifikace testovacích souborů.\\
\\
Program se bude spouštět příkazem:\\
\\
{\itshape classify.exe [spam] [spam-cnt] [ham] [ham-cnt] [test] [test-cnt] [out-file]}\\
\\
Symboly [spam], [ham] a [test] představují vzory jména vstupních souborů.\\
Symboly [spam-cnt], [ham-cnt] a [test-cnt] představují počty vstupních souborů.\\
Vstupní soubory mají následující pojmenování:
vzorN, kde N je celé číslo z intervalu [1; N]. Přípona všech vstupních souborů je .txt, přípona
není součástí vzoru. Váš program tedy může být během testování spuštěn například takto:\\
{\itshape. . . /classify.exe spam 10 ham 20 test 50 result.txt}\\
\\
Výsledkem činnosti programu bude textový soubor, který bude obsahovat seznam testovaných souborů
a jejich klasifikaci.
Pokud nebude na příkazové řádce uvedeno sedm argumentů, vypište chybové hlášení a stručný návod
k použití programu (v angličtině).\\
\\
Hotovou práci odevzdejte v jediném archivu typu ZIP prostřednictvím automatického odevzdávacího a
validačního systému. Archiv necht’ obsahuje všechny zdrojové soubory potřebné k přeložení programu,
makefile pro Windows i Linux (pro překlad v Linuxu připravte soubor pojmenovaný makefile a pro
Windows makefile.win) a dokumentaci ve formátu PDF vytvořenou v typografickém systému TEX,
resp. LATEX. Bude-li některá z částí chybět, kontrolní skript Vaši práci odmítne.\\
\\
Celé znění zadání práce je možné nalézt na následující url:\\
\url{http://www.kiv.zcu.cz/studies/predmety/pc/doc/work/sw2014-02.pdf}

%%%%%%%%%%%%%%%%%%%%%%%%%%%%%%%%%%%%%%%%%%%%%%%%%%%%%%%%%%%


%%%%%Analyza ulohy%%%%%%%%%%%%%%%%%%%%%%%%%%%%%%%%%%%%%%%%%
\chapter{Analýza úlohy}

Úkolem programu je pro zadané soubory určit, zda se jedná o nevyžádanou zprávu nebo ne. Jedná se o úlohu z oblasti strojového učení, kde se algoritmus na základě předložených vzorových dat naučí rozpoznávat data, která do té doby neznal.

\section{Algoritmus}
Dle doporučení v zadání jsem se rozhodl využít ke klasifikaci textu Naivní Bayesovský klasifikátor (NBK).\\
Lze použít i jiné algoritmy jako např.: Metodu podpůrných vektorů nebo Maximální entropie.

\subsection{Učení}
NBK vypočítává pravěpodobnost příslušnosti dokumentu ke třídě. Pro každou klasifikační třídu je nutné ypočítat:\\
P(c) - pravděpodobnost výskytu klasifikační třídy c v datech\\
P(word/c) - pravděpodobnost výskytu slova v klasifikační třídě

\subsection{Klasifikce}
Výsledná třída je argmax_{c_i\in C} {\left(\log(P(c_i)) + \sum_{k\in pozice}\log(P(word_k / c_i))\right)}\\
\\
Funkce argmax vrátí třídu c_i, pro kterou byla vypočtená hodnota nejvyšší.\\
pozice...všechny výskyty slov (obsažených ve slovníku) v testováném souboru\\
\\
Tento výpočet lze reprezentovat i jako: P(c_i) \times \prod_{k\in pozice} P(word_k / c_i)

\section{Datové struktury}
Rozhodl jsem se vytvořit slovník formou spojového seznamu, do kterého ukládám:
\begin{itemize}
	\item slovo
	\item celková četnost
	\item četnost ve spamu
	\item četnost v hamu
	\item pravděpodobnost příslušnosti ke spamu
	\item pravděpodobnost příslušnosti k hamu
	\item odkaz na další slovo v \uv{LinkedListu}
\end{itemize}
\\
Dále je potřeba pomocných proměnných pro vypočtení P(c) a několik pomocných proměných sloužící pro uložení celkových součtů dat.
%%%%%%%%%%%%%%%%%%%%%%%%%%%%%%%%%%%%%%%%%%%%%%%%%%%%%%%%%%%


%%%%%Popis implementace%%%%%%%%%%%%%%%%%%%%%%%%%%%%%%%%%%%%
\chapter{Popis implementace}

Iplementaci jsem rozdělil do tří bloků. První \uv{classify.c} s funkcí {\itmain()} vykonává funkci programu, \uv{load.c} zajišťuje načtení vstupních souborů a \uv{struct.c} se stará o slovník slov.

\section{classify.c}
\subsection{Main}
Nejprve otestuje správnost vstupních parametrů. Pokud jsou v nesprávném formátu vypíše nápovědu v angličtině.\\
Načte trénovací data a vypočte pravděpodobnost výskytu klasifikační třídy c v datech \uv{P(c)} a pravděpodobnost výskytu slova v klasifikační třídě \uv{P(word/c)}.\\
Zavolá funkci \uv{classify()}, která načte testované soubory a zařadí je do třídy spam/ham.

\subsection{Klasifikace}
\paragraph{Pro každý testovaný soubor:}
Načte testovaný soubor, pro každé slovo získá P(word/c) a spočte pravděpodobnost příslušnosti ke třídě spam nebo ham. Soubor spadá do třídy s větší pravděpodobností. Do výstupního souboru zapíše název testovaného souboru a jeho klasifikaci.

\section{load.c}
\subsection{Načtení trénovacích souborů}
Postupně načte všechny trénovací soubory spamu/hamu (musí se volat pro spam a ham zvlášť). Rozdělí řetězec slov na jednotlivá slova a uloží slova do slovníku.\\
Vrací počet všech slov, která byla načtena.
\subsection{Načtení testovaného souboru}
Načte jeden testovaný soubor a vrátí načtený řetězec.

\section{struct.c}
\subsection{Struct Dictionary}
Spojový seznam reprezentující slovník:
\begin{itemize}
	\item slovo
	\item celková četnost
	\item četnost ve spamu
	\item četnost v hamu
	\item pravděpodobnost příslušnosti ke spamu
	\item pravděpodobnost příslušnosti k hamu
	\item odkaz na další slovo v \uv{LinkedListu}
\end{itemize}

\subsection{Výpočet pravděpodobností pro jednotlivá slova}
Pro každě slovo ve slovníku spočte pravděpodobnost výskytu slova v klasifikační třídě:\\
P(word_k/c_i) = (n_k + 1) / (n + velikost\_slovníku)\\
n_k - počet výskytů slova (word_k) ve spamu/hamu (c_i)\\
n - počet slov patřících do spamu/hamu (c_i)

%%%%%%%%%%%%%%%%%%%%%%%%%%%%%%%%%%%%%%%%%%%%%%%%%%%%%%%%%%%


%%%%%Uzivatelska prirucka%%%%%%%%%%%%%%%%%%%%%%%%%%%%%%%%%%
\chapter{Uživatelská příručka}

Program je konzolová aplikace, která řadí testované soubory do kategorie spam/nespam dle předchozích vzorových souborů.

\section{Překlad programu}
Program lze přeložit pomocí nástroje make (v souborech \uv{makefile.win} pro Windows a  \uv{makefile} pro Linux jsou připravené instrukce ke kompilaci) nebo příkazem pro danný překladač (pro gcc: {\it gcc classify.c -o classify.exe}).


\section{Ovládání programu}
Program se ovládá výhradně parametry příkazové řádky při spuštění.\\
\\
{\it classify.exe [spam] [spam-cnt] [ham] [ham-cnt] [test] [test-cnt] [out-file]}\\
\\
Symboly [spam], [ham] a [test] představují vzory jména vstupních souborů.\\
Symboly [spam-cnt], [ham-cnt] a [test-cnt] představují počty vstupních souborů.\\
Symbol [out-file] představuje název výstupního souboru\\
Vstupní data jsou v adresáři \uv{/data}\\
\\
Příklad spuštění: {\it. . . /classify.exe spam 10 ham 20 test 50 result.txt}\\
\\
Po skončení programu je výsledek uložen v textovém souboru \uv{[out-file]} v adresáři s programem.\\
\\
Při zadání špatných parametrů se vypíše nápověda v angličtině.

%%%%%%%%%%%%%%%%%%%%%%%%%%%%%%%%%%%%%%%%%%%%%%%%%%%%%%%%%%%


%%%%%Zaver%%%%%%%%%%%%%%%%%%%%%%%%%%%%%%%%%%%%%%%%%%%%%%%%%
\chapter{Závěr}
Program při spuštění na malých datech (maximálně 399 trénovacích souborů) pracuje dle očekávání. Při velkých datech (400+ trénovacích souborů) na platformě \uv{Windows} sice probhěne a vrátí očekávané výsledky, ale při uvolňování paměti spojového seznamu reprezentující slovnik program zamrzne na 14062 položce seznamu (na dostupných trénovacích datech). Na platformě \uv{Linux} program pádá s chybovou hláškou {\it *** glibc detected *** ./classify.exe: malloc(): memory corruption: 0x080bf200 ***}  o chybném přístupu do paměti. Obě chyby jsou pravděpodobně způsebeny špatnými operacemi s pamětí, avšak nebyl jsem schopen najít chybu. Program valgrind nic nehlásí.\\
Dle trénovacích souborů program nedokáže rozeznat spam/ham na 100\%, vyskytují se drobné odchylky do 10\% v závislosti na počtu trénovacích souborů (od 100 trénovacích souborů spamu i hamu je program téměř stoprocentní). Tyto odchylky splňují stanovenou toleranci 10\% v zadání.\\
\\
Program je třeba doladit, aby dokázal pracovat i s většími daty. Dále by se dalo zlepšit ukládání načtených dat. Například při použití hash tabulky pro slovník by se zrychlilo ukládání i vyhledávání v datech.\\
\\
Semestrální práce byla pro mě zajímavá i přínosná z hlediska pochopení práce s pamětí a ukazateli, přestože se mi ji nepodařilo odevzdat na validátoru.

%%%%%%%%%%%%%%%%%%%%%%%%%%%%%%%%%%%%%%%%%%%%%%%%%%%%%%%%%%%


\end{document}

