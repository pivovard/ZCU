\setlength{\parskip}{1em}

\chapter*{Úvod}
\addcontentsline{toc}{chapter}{Úvod}

Předmětem této práce je vytvořit grafické uživatelské rozhraní tabletové aplikace pro jednotku intenzivní péče ve Fakultní nemocnici v Plzni. Tato aplikace je určena především pro zdravotní sestry. Na nemocničním pokoji bude k dispozici tablet s aplikací, kde zdravotní sestra bude mít k dispozici aktuální data o pacientech a bude do aplikace zanamenávat své provedené úkony.

Aplikace nahradí tištěnou formu medikačních záznamů a záznamů o pacientech. Umožní tak okamžitý přístup k datům v databázi a zefektivní proces přenosu a ukládání nových aktuálních dat. Aplikace zefektivní práci jak zdravotních sester, tak i doktorů. Každý provedený úkon zdravotní sestrou se okamžitě promítne do databáze a lékař ho uvidí na svém PC. Díky propojení dat s databází se předejde ručnímu přepisování, při kterém se zvyšuje chybovost.

Cílem je vytvořit jednoduché a intuitivní uživatelské rozhraní, které se podobá zavedeným postupům ve FN Plzeň. Vzorem pro vývoj tabletové aplikace je desktopová aplikace WinMedicalc vyvíjená plzeňskou firmou Medicalc software s.r.o. ve spolupráci s IT oddělením FN Plzeň.

Přestože nemocničních informačních systémů existuje celá řada, žádný nesplňuje požadavky FN Plzeň. Pouze jediný částečně vyhovoval, rozhodujícím faktorem pro nepořízení tohoto systému byly vysoké pořizovací náklady. Proto se nemocnice rozhodla pro vývoj systému na míru.

Aplikace má čtyři části: \emph{Medikace}, \emph{Bilance tekutin}, \emph{Hodinová bilance tekutin} a \emph{Invazivní přístupy}

Tabletovou aplikaci pro jednotku intenzivní péče jsem nazval pracovním názvem \emph{MediTab}.