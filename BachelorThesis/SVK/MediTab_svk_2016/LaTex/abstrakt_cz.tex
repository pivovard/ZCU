\documentclass{template_svk}

\usepackage[utf8]{inputenc}
\usepackage[czech]{babel}

\def\mtt#1{{\footnotesize\tt#1}}
\def\comm#1{\mtt{\char92#1}}

\begin{document}

\title{MediTab}

% Autoři
 \author{David Pivovar}{student bakalářského studijního programu Inženýrská informatika, obor Informatika, e-mail: pivovar@students.zcu.cz}

 \author{Daniel Švarc}{student bakalářského studijního programu Inženýrská informatika, obor Informatika, e-mail: svarc@students.zcu.cz}

\maketitle


\section{Úvod}

MediTab je tabletový informační systém určený pro zdravotní sestry pro správu záznamů o pacientech.

Aplikace doplňuje desktopovou aplikaci WinMedicalc (Medicalc software s. r. o.). Nahrazuje tištěnou formu záznamů o pacientech. Umožňuje tak okamžitý přístup k datům v databázi a zefektivňuje proces přenosu a ukládání nových aktuálních dat.


\section{Součásti aplikace MediTab:}

\begin{itemize}
	\item Medikační karta
	\item Denní bilance tekutin
	\item Hodinová bilance tekutin
	\item Invazivní přístupy
	\item Fyziologie
\end{itemize}


\subsection{Medikační karta}

Záložka medikační karty nahrazuje tištěnou medikační kartu. Jedná se o tabulku s názvy léku, předepsaným dávkováním a jednotlivými hodinami dávkování. Zdravotní sestry zde zaznamenávají jednotlivá podání daného léku a mohou je i editovat.


\subsection{Denní bilance tekutin}

V záložce denní bilance tekutin se zaznamenávají veškeré příjmy a výdaje jednotlivých tekutin za celý den. Hodnotu je možné zadat jako celkový součet, nebo jen novou hodnotu, která se přičte k původní.


\subsection{Hodinová bilance tekutin}

Hodinová bilance tekutin je totožná s denní bilancí tekutin. Navíc je rozdělena na jednotlivé hodiny pro sledování vývoje příjmu a výdajů tekutin.


\subsection{Invazivní přístupy}

Pacient může mít zavedeny invazivní přístupy (katetry, drény). Ty se zobrazují v záložce invazivních přístupů, kde je může zdravotní sestra editovat, zadat požadavek na výměnu, nebo smazat pokud byl invazivní přístup z pacienta odstraněn.


\subsection{Fyziologie}

V této záložce se zaznamenávají životní funkce pacientů (teplota, tlak, dýchání).


\section{Technické parametry}

Aplikace běží na systému Microsoft Windows 8.1 64bit. Je vyvíjená na platformě Microsoft .NET Framework 4.5 v jazyce C\#.

Pro provoz aplikace v nemocnici byly vybrány tablety HP ElitePad 1000G2 Healthcare Tablet, které jsou schváleny pro použití v nemoničním prostředí. Tyto tablety mají antibakteriální povrchovou úpravu a odolnější konstrukci.

Aplikace je kompatibilní s informačním systémem FN Plzeň a bude doplňovat program WinMedicalc. Systémová integrita mezi oběmi aplikacemi není, pouze jsou využity některé funkční prvky WinMedicalcu.

\acknowledgement{Projekt je podporován SIS FN Plzeň.}

\end{document}
