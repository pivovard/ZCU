\setlength{\parskip}{1em}

\chapter{Advanced Encryption Standard}

Advanced Encryption Standard (AES) je standardizovaný algoritmus používaný k šifrování dat v informatice. Je založen na Rijndael agoritmu. Jedná se o symetrickou blokovou šifru šifrující i dešifrující stejným klíčem data rozdělená do bloků pevně dané délky.

AES používá k šifrování substituce i permutace. Pracuje s maticí bytů 4x4 označovanou jako stav. Má pevně danou velikost bloku na 128 bitů a velikost klíče na 128, 192 nebo 256 bitů. Velikost klíče určuje počet cyklů algoritmu.

\begin{itemize}
	\item 128-bit klíč: 10 cyklů
	\item 192-bit klíč: 12 cyklů
	\item 256-bit klič: 14 cyklů
\end{itemize}


\section{Popis algoritmu}

\begin{enumerate}
	\item Expanze klíče - podklíče jsou odvozeny z klíče pomocí seznamu Rijndael klíčů (Rijndael key schedule)
	\item Inicializační část
	\begin{enumerate}
		\item Přidání podklíče - každý byte stavu je zkombinován s podklíčem za pomoci operace xor nad všemi bity
	\end{enumerate}
	\item Iterace
	\begin{enumerate}
		\item Záměna bytů - každý byte je nahrazen jiným podle substituční tabulky
		\item Prohození řádků - každý řádek stavu je postupně posunut o určitý počet kroků
		\item Kombinování sloupců - zkombinuje čtyři byty v každém sloupci
		\item Přidání podklíče
	\end{enumerate}
	\item Závěrečná část
	\begin{enumerate}
		\item Záměna bytů
		\item Prohození řádků
		\item Přidání podklíče
	\end{enumerate}
\end{enumerate}